%!TeX root = ../index.tex

\section{Introduction}\label{sec:introduction}

\section{Reactive \acrlongpl{is}}

\citeauthor{boner_reactive_2014} describe how modern \glspl{is} of different domains all exhibit similar qualities.
\citetitle{boner_reactive_2014} summarizes those qualities and dubs the systems which feature them \enquote{reactive} \parencite{boner_reactive_2014}.
Put into the context of the quality dimensions defined by the \citetitle{iso_25010_2011} standard,
the reactive qualities cover only \enquote{Performance Efficiency} and \enquote{Reliability}.
While the other quality dimensions of the \citeauthor{iso_25010_2011} standard are certainly also important, this work focuses on the two mentioned before.

To quickly summarize \cite{boner_reactive_2014}, reactive systems are:

\begin{description}
  \item[Responsive] The system has low upper ceilings for its response times and meets these consistently.
  \item[Resilient] The system remains responsive in the face of failure.
  \item[Elastic] The system remains responsive under varying workload.
\end{description}

These qualities cover \enquote{reactive} systems from a technological perspective.
However, there is also a business perspective to the term.
Reactive systems also need to be flexible.
They need to be able to respond and change quickly according to changing requirements.
For example due to changing customer or market needs.
\citeauthor{beck2001agile} describe this quality as part of their \citetitle{beck2001agile} \parencite{beck2001agile}.
This work will refer to this quality as \emph{flexibility} from here on.

Different design and work patterns have emerged to create reactive systems.
\cite{boner_reactive_2014} references one directly in the fourth quality of reactive systems: Message-driven.
Message-driven systems \enquote{rely on asynchronous message-passing to
establish a boundary between components that ensures loose coupling, isolation,
location transparency, and provides the means to delegate errors as messages.} \parencite{boner_reactive_2014}
This description of a message-driven system assumes that it is a set of separate components.
While this does not necessarily reference microservice architecture---another pattern for building reactive systems---the pattern definitely fits the description.

\subsection{Microservice Architecture}



\subsection{\acrlong{mom}}

\subsection{Apache Kafka}

The LinkedIn employees \citeauthor{kreps_kafka_2011} introduced their new \enquote{Event Streaming Platform} \emph{Kafka} in \citeyear{kreps_kafka_2011} \parencite{kreps_kafka_2011}.
Over a decade later, the project has since joined the Apache foundation and \enquote{[m]ore than 80\% of all Fortune 100 companies trust, and use [it]} \parencite{apache_software_foundation_apache_nodate}.

\citeauthor{kreps_kafka_2011} describe Kafka as \enquote{[\ldots] a novel messaging system for log processing [\ldots] that combines the benefits of traditional log aggregators and messaging systems.} \parencite{kreps_kafka_2011}.
The solution has an \gls{api} that is similar to traditional \gls{mom}, yet it persists messages in an append-only log structure on the hard drive.
This detail allows Kafka to not only stream data in near real-time but persist large amounts of it for online analytical processing.
In addition to that, Kafka features partitioning and replication mechanisms which make it highly available and scalable. \parencite{kreps_kafka_2011}

Kafka is not only useful for processing the kinds of log data that are proposed by \citeauthor{kreps_kafka_2011}: activity (e.~g. user logins, clicks, \enquote{likes}, \ldots) and operational data (e.~g. \gls{cpu} or disk utilization, \gls{http} requests, \ldots) \parencite{kreps_kafka_2011}.
\citeauthor{stopford_designing_2018}, for example, outlines how the entire state management and internal communication of a business application can be based on Apache Kafka.
His design implements concepts from the \gls{ddd} community like Event Sourcing \parencite{fowler_event_sourcing_2005} and \gls{cqrs} \parencite{fowler_cqrs_2011} with Apache Kafka's log-based messaging at the center \parencite{stopford_designing_2018}.
The potential use cases of Apache Kafka are therefore plentiful.

\section{Schemas}

\subsection{Schema Evolution}

\subsection{Schema Revolution}

\subsection{Apache Avro}

Apache Avro provides the ability to define rich data structures and interfaces for \glspl{rpc} as schema documents.
On top of that, it offers an efficient binary data format for serialization.
Listing \ref{lst:avro-schema-person} shows an example for an Avro schema describing a customer entity.
\parencite{apache_software_foundation_apache_2021}

\begin{listing}[H]
  \inputminted{json}{assets/src/Customer.avsc}
  \caption{Simplified Avro Schema of a Customer Entity}\label{lst:avro-schema-person}
\end{listing}

Avro's design assumes that applications which (de-)serialize data have access to the data's schema.
That is one of the reasons why the format is so efficient.
It does not need to include type information in the serialized data.
Avro's support for schema evolution builds upon this assumption.
If an application attempts to de-serialize data encoded in Avro binary, it requires both the schema with which the data was written and the schema which the application understands.
These schemas are called the \emph{writer} and the \emph{reader} schema respectively.
In the simplest case, these schemas might be equal.
Nevertheless, Avro also supports de-serializing data with a reader schema different from the writer schema.
Although, the de-serialization will only succeed if both schemas are still \emph{compatible}.
\parencite{apache_software_foundation_apache_2021}

If, for example, the reader schema had an additional field without a default value.
It would be an incompatible change because Avro would not be able to infer a value for the additional field.
On the other hand, if the field had a default value, the de-serialization would succeed.
\citeauthor{kreps_kafka_2011} chose Apache Avro as the serialization protocol for their message payloads due to its efficiency and schema evolution capabilities \parencite{kreps_kafka_2011}. 

\section{Schema Management}

\subsection{Schemas In Distributed Systems}

\subsection{Schema Management Solutions}

\section{Overview of Schema Management Solutions}
