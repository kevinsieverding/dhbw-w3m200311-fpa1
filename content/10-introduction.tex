%!TeX root = ../index.tex

\section{Introduction}\label{sec:introduction}

Successful, modern \gls{is} are \emph{reactive}, meaning they are responsive, resilient, elastic and message-driven \parencite{boner_reactive_2014}.
The microservice architectural pattern in combination with \gls{mom} is a popular means to design reactive systems.\footnote{\cites()(85){loukides_microservice_adoption_2020}{fowler_microservices_2014}{boner_reactive_2014}{richardson_microservices_2019}.}
Apache Kafka is a popular \gls{mom} that was originally intended for handling log data (clicks, likes, HTTP requests, \ldots) \parencite{kreps_kafka_2011} but can also be used to create reactive business applications \parencite{stopford_designing_2018}.

\citeauthor[]{kreps_kafka_2011} describe that their Kafka setup at LinkedIn uses Apache Avro to serialize messages for its compact binary format and to enforce compatibility between their services.
They also mention a \enquote{lightweight schema registry service} to enable message consumers to look up schemas based on IDs.
That is the extent of the description \citeauthor[]{kreps_kafka_2011} give.
They do not provide any background on why they use Avro schemas and a schema registry, nor do they provide details on their benefits.
\parencite{kreps_kafka_2011}

Additionally, other works that present \gls{is} designs using Kafka also mention using a schema registry.
Many describe that they use the Confluent Schema Registry\footnote{\cites{radchenko_micro-workflows_2018}{ranjan_radar-base_2019}{korhonen_using_2019}{auer_distributed_2017}{dessalegn_muruts_multi-tenant_2016}}
while some only mention that they use a schema registry without specifying further.\footnote{\cites{g_b_high_2021}{muller_iot_2017}}

That leaves the question: "What benefits do schemas and schema registries provide to a distributed system using \gls{mom}?"
This work attempts to answer that question by examining an example \gls{is} architecture that follows the microservice pattern and uses Kafka.
It explores how Avro schemas and a schema registry can be introduced to the architecture and judges their benefits based on the qualities of a reactive system laid out by \cite{boner_reactive_2014}.

The sections \ref{sec:reactive-is} and \ref{sec:schemas-in-reactive-is} cover the basics of reactive systems, microservices, Kafka, schemas, and Avro before introducing the example architecture in \ref{sec:web-shop}.
That architecture serves as a basis for identifying the benefits of schemas for reactive systems in \ref{sec:schema-benefits} and the exploration of how they can be introduced to reactive systems in \ref{sec:introducing-schemas}.
Section \ref{sec:schema-management} continues this exploration for schema registries, culminating in a general definition of schema management.
Lastly, section \ref{sec:conclusion} reflects on the work and touches on topics for future undertakings.