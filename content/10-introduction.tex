%!TeX root = ../index.tex

\section{Introduction}\label{sec:introduction}

My team and I working at SAP chose Apache Kafka as the central \gls{mom} of our new application.
It is supposed to facilitate the communication between the various microservices that make up our application.
We also use Apache Avro do define schemas for the payload's or our messages and to serialize them.
It soon became apparent to us, that if we wanted fully exploit Avro's capabilities for schema evolution in our distributed system, we needed a dedicated component to manage and serve our schemas.
We discovered our requirement for schema management.

As we began searching for appropriate software solutions to address this requirement, two things became pretty clear to us: Firstly, the Confluent Schema Registry is the de facto standard solution for the management of message schemas in combination with Apache Kafka. Secondly, despite the dominance of Confluent's product, a variety of available solutions exists.

Choosing a schema management solution is difficult for a variety of reasons. We experienced the following:

\begin{enumerate}
  \item The problem domain of schema management is poorly defined.
  \item The characteristics of a schema management solution are poorly defined.
  \item The field of schema management solutions is dominated by the de facto standard. The Confluent Schema Registry.
\end{enumerate}

The goal of this work is to support the decision process of choosing a schema management solution.
To achieve this, both the problem domain of schema management as well as the market for schema management solutions are examined.
Based on this, a definition of schema management is deduced, as well as the characteristics of a schema management solution.
These characteristics are then used to identify and compare the schema management solutions currently on the market.

\todo{structure}
