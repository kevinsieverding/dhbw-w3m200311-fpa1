%!TeX root = ../index.tex

\section{Introduction}\label{sec:introduction}

My team and I, working at SAP, chose Apache Kafka as the central \gls{mom} of our new application.
It is supposed to facilitate the communication between the various microservices that make up our application.
We also use Apache Avro to define schemas for our messages' payloads and serialize them.
It soon became apparent that the distributed nature of our system requires additional effort when using schemas.
If we wanted to fully exploit Avro's capabilities for schema evolution, we needed a dedicated component to manage and serve our schemas.
We discovered our requirement for schema management.

As we began searching for appropriate software solutions to address this requirement, two things became pretty clear: Firstly, the Confluent Schema Registry is the de facto standard solution for managing message schemas in combination with Apache Kafka. Secondly, despite the dominance of Confluent's product, a variety of available solutions exists.

Choosing a schema management solution is difficult for a variety of reasons. We experienced the following:

\begin{enumerate}
  \item The problem domain of schema management is poorly defined.
  \item The characteristics of a schema management solution are unclear.
  \item The Confluent Schema Registry dominates the market for schema management solutions as the de facto standard.
\end{enumerate}
\todo{citations?}

The goal of this work is to support the decision-making process of choosing a schema management solution.
The work examines the problem domain of schema management and the market for schema management solutions.
Subsequently, it deduces a definition of schema management and the characteristics of a schema management solution.
The work then identifies and compares the schema management solutions currently on the market using these characteristics.
It thereby addresses all three of the problems mentioned above.

\todo{structure}
