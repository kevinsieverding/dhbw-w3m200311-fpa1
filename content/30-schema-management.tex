%!TeX root = ../index.tex

\section{Schema Management}\label{sec:schema-management}

This section examines the problem domain of schema management by analyzing an example use case and parallel use cases from academic literature.
Before though, it provides essential information regarding the two prominent technologies of these use cases: Apache Kafka and Apache Avro.
Building on the analysis, the subsequent text deduces a definition of schema management and the characteristics of a schema management solution.

\subsection{Messaging Technologies}

\subsubsection{Apache Kafka}

The LinkedIn employees \citeauthor{kreps_kafka_2011} introduced Kafka in \citeyear{kreps_kafka_2011}.
Over a decade later, the project has since joined the Apache foundation and \enquote{[m]ore than 80\% of all Fortune 100 companies trust, and use [it].} \parencite{noauthor_apache_nodate}.

\citeauthor{kreps_kafka_2011} describe Kafka as \enquote{[\ldots] a novel messaging system for log processing [\ldots] that combines the benefits of traditional log aggregators and messaging systems.} \parencite{kreps_kafka_2011}.
The solution has an \gls{api} that is similar to traditional \gls{mom}, yet it persists messages in an append-only log structure on the hard disk.
This detail allows Kafka to not only stream data in near real-time but persist large amounts of it for online analytical processing.
In addition to that, Kafka features partitioning and replication mechanisms which make it highly available and scalable. \parencite{kreps_kafka_2011}

\subsubsection{Apache Avro}

\citeauthor{kreps_kafka_2011} describe that they chose Apache Avro as the serialization protocol for its efficiency and schema evolution capabilities. 
Avro provides the ability to define rich data structures and interfaces for \glspl{rpc} in schema documents.
Furthermore, it offers an efficient binary data format for serialization.
Avro's design assumes that applications that (de)-serialize data have access to the data's schema.
That is one of the reasons why the format is so efficient since it does not need to include type information.


\subsection{Schema Management Use Cases}

\subsection{Definition of Schema Management}

\subsection{Characteristics of Schema Management Solutions}
